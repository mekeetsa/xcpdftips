% \iffalse      THIS IS A META-COMMENT
%<*dtx>
\ProvidesFile
%========================================================================
                            {XCPDFTIPS.DTX}
%========================================================================
%</dtx>
%% Copyright (c) 2019 Mikica Kocic
%
% This program is free software: you can redistribute it and/or modify
% it under the terms of the GNU General Public License as published by
% the Free Software Foundation, either version 3 of the License, or
% (at your option) any later version.
%
% This program is distributed in the hope that it will be useful,
% but WITHOUT ANY WARRANTY; without even the implied warranty of
% MERCHANTABILITY or FITNESS FOR A PARTICULAR PURPOSE.  See the
% GNU General Public License for more details.
%
% You should have received a copy of the GNU General Public License
% along with this program.  If not, see <https://www.gnu.org/licenses/>.
%
% This is a contributed file to the LaTeX2e system.
% -------------------------------------------------
%         This is a LaTeX package to do citations with PDF tooltips.
% Installation:
%    LaTeX this file: creates docstrip installation file xcpdftips.ins
%                         AND the LaTeX documentation
%    (La)TeX xcpdftips.ins: creates package file xcpdftips.sty, and optionally
%                         the documentation driver xcpdftips.drv
%    (xcpdftips.ins and xcpdftips.drv may be edited as needed)
% Docstrip options available:
%        package - to produce a (LaTeX2e) package .sty file
%        driver  - to produce a driver file to print the documentation
%--------------------------------------------------------------------------
%
%  *** Identify the package file:-
%<package>
%<package>\NeedsTeXFormat{LaTeX2e}
%<package>
%<package>\ProvidesPackage{xcpdftips}
%<package>[2019/03/10 xcpdftips.sty v1.0 - (c) 2019 Mikica Kocic]
%<package>
%<package>\RequirePackage{bibentry}
%<package>\RequirePackage{pdfcomment}
%<package>\RequirePackage{xparse}
%<package>\RequirePackage{etoolbox}
%
%  *** Identify the driver file:-
%
%<driver>\NeedsTeXFormat{LaTeX2e}
%<driver>\ProvidesFile{xcpdftips.drv}
%
%  *** The DATE, VERSION, and other INFO
%
%\fi
%\ProvidesFile{xcpdftips}[2019/03/10 1.0 (MK)]
% \changes{1.0}{2019 Mar 10}{Initial version}
% \changes{1.1}{2019 Mar 15}{Robusitified \\xpdfcite}
%
% \CheckSum{130}
% \CharacterTable
%  {Upper-case    \A\B\C\D\E\F\G\H\I\J\K\L\M\N\O\P\Q\R\S\T\U\V\W\X\Y\Z
%   Lower-case    \a\b\c\d\e\f\g\h\i\j\k\l\m\n\o\p\q\r\s\t\u\v\w\x\y\z
%   Digits        \0\1\2\3\4\5\6\7\8\9
%   Exclamation   \!     Double quote  \"     Hash (number) \#
%   Dollar        \$     Percent       \%     Ampersand     \&
%   Acute accent  \'     Left paren    \(     Right paren   \)
%   Asterisk      \*     Plus          \+     Comma         \,
%   Minus         \-     Point         \.     Solidus       \/
%   Colon         \:     Semicolon     \;     Less than     \<
%   Equals        \=     Greater than  \>     Question mark \?
%   Commercial at \@     Left bracket  \[     Backslash     \\
%   Right bracket \]     Circumflex    \^     Underscore    \_
%   Grave accent  \`     Left brace    \{     Vertical bar  \|
%   Right brace   \}     Tilde         \~}
%
% \iffalse
%<*install>
%
%^^A =============================================
%^^A    Here is the docstrip installation file
%^^A    It is written on first LaTeX run if it
%^^A    does not already exist
%^^A =============================================
%
%%%%%%%%%%%%%%%%%%%%%%%%%%%%%%%%%%%%%%%%%%%%%%%%%%%%%%%%%%%%%%%%%%%%%%%%%%%%%%%%
\begin{filecontents*}{xcpdftips.ins}
%
% File: xcpdftips.ins
%
% Copyright (c) 2019 Mikica Kocic
%
% This program is free software: you can redistribute it and/or modify
% it under the terms of the GNU General Public License as published by
% the Free Software Foundation, either version 3 of the License, or
% (at your option) any later version.
%
% This program is distributed in the hope that it will be useful,
% but WITHOUT ANY WARRANTY; without even the implied warranty of
% MERCHANTABILITY or FITNESS FOR A PARTICULAR PURPOSE.  See the
% GNU General Public License for more details.
%
% You should have received a copy of the GNU General Public License
% along with this program.  If not, see <https://www.gnu.org/licenses/>.
%
% It is an installation file for extracting package and driver
% files from the original source file. Simply process it under
% TeX or LaTeX. It works with Docstrip versions before and after
% December 1995.

\def\batchfile{xcpdftips.ins}
\input docstrip

\preamble

=============================================
IMPORTANT NOTICE:

This program is free software: you can redistribute it and/or modify
it under the terms of the GNU General Public License as published by
the Free Software Foundation, either version 3 of the License, or
(at your option) any later version.

This program is distributed in the hope that it will be useful,
but WITHOUT ANY WARRANTY; without even the implied warranty of
MERCHANTABILITY or FITNESS FOR A PARTICULAR PURPOSE.  See the
GNU General Public License for more details.

You should have received a copy of the GNU General Public License
along with this program.  If not, see <https://www.gnu.org/licenses/>.

This is a generated file.
It may not be distributed without the original source file \inFileName.

Full documentation can be obtained by LaTeXing that original file.
Only a few abbreviated comments remain here to describe the usage.
=============================================

\endpreamble
\postamble

<<<<< End of generated file <<<<<<

\endpostamble
\keepsilent

% Docstrip before Dec 95 does not have \generate syntax, nor
%   \declarepreamble. Must redefine them. The \generateFile called
%   for each output file individually.
% Docstrip before Dec 96 cannot interprete multiline \if..\fi
%   Thus for maximum compatibility, have only one-line conditionals

\let\oldDS F\relax
\expandafter\ifx\csname generate\endcsname\relax \let\oldDS T\relax\fi
\if\oldDS T  \def\declarepreamble#1{\preamble}\fi
\if\oldDS T  \def\declarepostamble#1{\postamble}\fi
\if\oldDS T  \generateFile{xcpdftips.sty}{f}{\from{xcpdftips.dtx}{package}} \fi

\declarepreamble\driver
============================================
This is the driver file to produce the LaTeX documentation
from the original source file \inFileName.

Make changes to it as needed. (Never change the file \inFileName!)
============================================
\endpreamble

\declarepostamble\driverq

End of documentation driver file.
\endpostamble

\ifx\oldDS T \generateFile{xcpdftips.drv}{f}{\from{xcpdftips.dtx}{driver}}\fi

\ifx\oldDS T \let\askforoverwritefalse\relax\def\generate#1{}\fi

\askforoverwritefalse
\generate{\file{xcpdftips.sty}{\from{xcpdftips.dtx}{package}}
          \file{xcpdftips.drv}{\usepreamble\driver\usepostamble\driverq
                           \from{xcpdftips.dtx}{driver}}
         }

\obeyspaces
\Msg{********************************************}%
\Msg{* For documentation, process xcpdftips.dtx *}%
\Msg{*    or the driver file      xcpdftips.drv *}%
\Msg{********************************************}
\end{filecontents*}
%</install>
%<*driver>
\documentclass[a4paper]{ltxdoc}
%<driver>%\documentclass[twoside]{ltxdoc}
%<driver>%\documentclass[a4paper]{ltxdoc}
%<driver>%\documentclass[twoside,a4paper]{ltxdoc}
\raggedbottom

 %** To include the detailed explanation of the coding, comment out
 %**   the next line
% \OnlyDescription

 %** To produce a command index: add the following line for one run,
 %**   then run  makeindex -s gind.ist xcpdftips
 %**   and reprocess, with or without this line (much faster without)
%<driver>% \EnableCrossrefs\CodelineIndex

 %** To produce a change history: add the following line for one run,
 %**   then run  makeindex -s gglo.ist -o xcpdftips.gls xcpdftips.glo
 %**   and reprocess, with or without this line (faster without)
%<driver>% \RecordChanges

\DisableCrossrefs %May stay; zapped by \EnableCrossrefs
\CodelineNumbered %May stay

%%%%%%%%%%%%%%%%%%%%%%%%%%%%%%%%%%%%%%%%%%%%%%%%%%%%%%%%%%%%%%%%%%%%%%%%%%%%%%%%
\begin{document}
   \DocInput{xcpdftips.dtx}
\end{document}
%</driver>
%\fi
%
% \DoNotIndex{\begin,\CodelineIndex,\CodelineNumbered,\def,\DisableCrossrefs}
% \DoNotIndex{\DocInput,\documentclass,\EnableCrossrefs,\end,\GetFileInfo}
% \DoNotIndex{\NeedsTeXFormat,\OnlyDescription,\RecordChanges,\usepackage}
% \DoNotIndex{\ProvidesClass,\ProvidesPackage,\ProvidesFile,\RequirePackage}
% \DoNotIndex{\filename,\fileversion,\filedate,\let}
% \DoNotIndex{\@listctr,\@nameuse,\csname,\else,\endcsname,\expandafter}
% \DoNotIndex{\gdef,\global,\if,\item,\newcommand,\nobibliography}
% \DoNotIndex{\par,\providecommand,\relax,\renewcommand,\renewenvironment}
% \DoNotIndex{\stepcounter,\usecounter,\nocite,\fi}
% \DoNotIndex{\@fileswfalse,\@gobble,\@ifstar,\@unexpandable@protect}
% \DoNotIndex{\AtBeginDocument,\AtEndDocument,\begingroup,\endgroup}
% \DoNotIndex{\frenchspacing,\MessageBreak,\newif,\PackageWarningNoLine}
% \DoNotIndex{\protect,\string,\xdef,\ifx,\texttt,\@biblabel,\bibitem}
%
% \setcounter{IndexColumns}{2}
% \setlength{\IndexMin}{10cm}
% \setcounter{StandardModuleDepth}{1}
%
% \GetFileInfo{xcpdftips}
%
% \title{\bfseries Citations with PDF tooltips}
%
% \author{Mikica Kocic}
%
% \date{This paper describes package \texttt{\filename}\\
%       version \fileversion{} from \filedate
%  }
%
% \maketitle
%
% \pagestyle{myheadings}
% \markboth{M. Kocic}{PDF tooltips from natbib citations}
%
%^^A In order to keep all marginal notes on the one (left) side:
%^^A (otherwise they switch sides disasterously with twoside option)
% \makeatletter \@mparswitchfalse \makeatother
%
%\iffalse
%<*package>
%
% PDF tooltips from natbib citations
%
%-----------------------------------------------------------
% See documentation in the source .dtx file for more details.
%</package>
%\fi
%
% \section{Introduction}
%
% This package allows one to be able to do \texttt{natbib} citations
% with PDF tooltips.
%
% \section{Invoking the Package}
%
% The macros in this package are included in the main document
% with the |\usepackage| command of \LaTeXe,
% \begin{quote}
% |\documentclass[..]{...}|\\
% |\usepackage{|\texttt{\filename}|}|
% \end{quote}
%
% \section{Usage}
%
% \newcommand\btx{\textsc{Bib}\TeX}
%
% This package must be used with \btx{} and \texttt{natbib}, not with a 
% hand-written \texttt{thebibliography} environment.
% More precisely, there must be a \texttt{.bbl} file external to the \LaTeX\
% file; whether this is written by hand or by \btx\ is unimportant.
% \vspace{2ex}
%
% \DescribeMacro{\xpdfcite}
% This is a replacement for \texttt{natbib}'s |\cite| macro. 
% Usage is the same:
%
% \begin{quote}
%      |\xpdfcite|\marg{key(s)}
% \end{quote}
%
% \noindent Similarly to |\cite|, the command |\xpdfcite| may take one or two
% optional arguments to add some text before and after the citation.
% \vspace{2ex}
%
% \noindent It is also possible to replace |\cite|:
%
% \begin{quote}
%      |\usepackage{xcpdftips}| \\
%      |\let\cite\xpdfcite|
%      |\robustify{\cite}|
% \end{quote}
%
% The last command |\robustify{\cite}| is needed if you wish to use
% |\cite|, for instance, in captions.
%
% \section{Caveats}
%
% The \texttt{\filename} package will work with \texttt{natbib} with its
% native |\bibitem| format, and with standard \LaTeX. Nothing else can be
% guaranteed.
% It will also work with \texttt{url} package. 
%
% \StopEventually{\PrintIndex\PrintChanges}
%
% \section{Options with \texttt{docstrip}}
%
% The source \texttt{.dtx} file is meant to be processed with
% \texttt{docstrip}, for which a number of options are available:
%
%%%%%%%%%%%%%%%%%%%%%%%%%%%%%%%%%%%%%%%%%%%%%%%%%%%%%%%%%%%%%%%%%%%%%%%%%%%%%%%%
% \begin{description}
%
% \item[\ttfamily package] to produce a \texttt{.sty} package file with most
%     comments removed;
%
% \item[\ttfamily driver] to produce a driver \texttt{.drv} file that will
%     print out the documentation under \LaTeXe. The documentation cannot
%     be printed under \LaTeX~2.09.
%
% \end{description}
%
% The source file \texttt{\filename.dtx} is itself a driver file and can
% be processed directly by \LaTeXe.
%
% \section{The Coding}
%
% This section presents and explains the actual coding of the macros.
% It is nested between |%<*package>| and |%</package>|, which
% are indicators to \texttt{docstrip} that this coding belongs to the package
% file.
%
%%%%%%%%%%%%%%%%%%%%%%%%%%%%%%%%%%%%%%%%%%%%%%%%%%%%%%%%%%%%%%%%%%%%%%%%%%%%%%%%
% \begin{macro}{\XC@enumeratetips}
%
% The macro |\XC@enumeratetips| gets \texttt{bibentry} for each key from 
% the list of citations. The output is stored into |\XC@tips|, which can
% be directly used as a tooltip text in |\pdftooltip|.
%
%    \begin{macrocode}
%<*package>

\ExplSyntaxOn

\NewDocumentCommand{ \XC@enumeratetips }%
{ > { \SplitList , } m }%
{%
  \global\undef\XC@tips%
  \undef\XC@tipsPre%
  \tl_map_inline:nn {#1}%
  {%
    \ifx\XC@tips\undefined%
      \global\def\XC@tips{}%
      \gappto{\XC@tips}{\@nameuse{BR@r@##1\@extra@b@citeb}}%
    \else%
      \def\XC@tipsPre{\textbullet\ \ }
      \gappto{\XC@tips}{,\textCR\textbullet\ \ \@nameuse{BR@r@##1\@extra@b@citeb}}%
    \fi%
  }%
  \ifx\XC@tipsPre\undefined\else%
    \gpreto{\XC@tips}{\XC@tipsPre}%
  \fi%
}

\ExplSyntaxOff

%    \end{macrocode}
% \end{macro}
%
%%%%%%%%%%%%%%%%%%%%%%%%%%%%%%%%%%%%%%%%%%%%%%%%%%%%%%%%%%%%%%%%%%%%%%%%%%%%%%%%
% \begin{macro}{\xcsetauthor}
%
% This commands sets the author used for pdf comments (default: {}).
%
%    \begin{macrocode}

\gdef\XC@opt@author{{}}
\newcommand{\xcsetauthor}[1]{\gdef\XC@opt@author{#1}}

%    \end{macrocode}
% \end{macro}
%
%%%%%%%%%%%%%%%%%%%%%%%%%%%%%%%%%%%%%%%%%%%%%%%%%%%%%%%%%%%%%%%%%%%%%%%%%%%%%%%%
% \begin{macro}{\xcsetmarkup}
%
% This commands sets the markup used for pdf comments (default: Underline).
%
%    \begin{macrocode}

\gdef\XC@opt@markup{Underline}
\newcommand{\xcsetmarkup}[1]{\gdef\XC@opt@markup{#1}}

%    \end{macrocode}
% \end{macro}
%
%%%%%%%%%%%%%%%%%%%%%%%%%%%%%%%%%%%%%%%%%%%%%%%%%%%%%%%%%%%%%%%%%%%%%%%%%%%%%%%%
% \begin{macro}{\xcsetcolor}
%
% This commands sets the color used for pdf comments (default: yellow).
%
%    \begin{macrocode}

\gdef\XC@opt@color{1 1 0}
\newcommand{\xcsetcolor}[1]{\gdef\XC@opt@color{#1}}

%    \end{macrocode}
% \end{macro}
%
%%%%%%%%%%%%%%%%%%%%%%%%%%%%%%%%%%%%%%%%%%%%%%%%%%%%%%%%%%%%%%%%%%%%%%%%%%%%%%%%
% \begin{macro}{\XC@opt@opacity}
%
% This commands sets the opacity used for pdf comments (default: 0).
%
%    \begin{macrocode}

\gdef\XC@opt@opacity{0}
\newcommand{\xcsetopacity}[1]{\gdef\XC@opt@opacity{#1}}

%    \end{macrocode}
% \end{macro}
%
%%%%%%%%%%%%%%%%%%%%%%%%%%%%%%%%%%%%%%%%%%%%%%%%%%%%%%%%%%%%%%%%%%%%%%%%%%%%%%%%
% \begin{macro}{\XC@citetp}
%
% This macro is in fact |\xpdfcite|. \\
% It is a wrapper for |\XC@@citetp| to handle variable number of arguments.
%
%    \begin{macrocode}

\newcommand\XC@citetp{\@ifnextchar[{\XC@@citetp}{\XC@@citetp[]}}

%    \end{macrocode}
% \end{macro}
%
%%%%%%%%%%%%%%%%%%%%%%%%%%%%%%%%%%%%%%%%%%%%%%%%%%%%%%%%%%%%%%%%%%%%%%%%%%%%%%%%
% \begin{macro}{\XC@@citetp}
%
% This macro is called from |\xpdfcite|. \\
% It is a wrapper for |\XC@citex| to handle variable number of arguments.
%
%    \begin{macrocode}

\newcommand\XC@@citetp{}

\def\XC@@citetp[#1]{\@ifnextchar[{\XC@citex[#1]}{\XC@citex[][#1]}}

%    \end{macrocode}
% \end{macro}
%
%%%%%%%%%%%%%%%%%%%%%%%%%%%%%%%%%%%%%%%%%%%%%%%%%%%%%%%%%%%%%%%%%%%%%%%%%%%%%%%%
% \begin{macro}{\XC@citex}
%
% This is an internal macro that does the job of |\xpdfcite|.
% It combines |\citep| with |\pdftooltip| and |\pdfmarkupcomment|. 
%
%    \begin{macrocode}

\newcommand\XC@citex{}

\def\XC@citex[#1][#2]#3%
  {%{\protect\NoHyper%
    \ifx\ocgtooltip\undefined%
      \XC@enumeratetips{#3}%
      \pdftooltip{\XC@oldcite[#1][#2]{#3}}{\XC@tips}%
      \pdfmarkupcomment[%
        author=\XC@opt@author,%
        markup=\XC@opt@markup,%
        color=\XC@opt@color,%
        opacity=\XC@opt@opacity,%
      ]{\vphantom{.}}{\XC@tips}%
    \else%
      \let\textCR\par%
      \XC@enumeratetips{#3}%
      \ocgtooltip{%
        {{\protect\NoHyper\XC@oldcite[#1][#2]{#3}\protect\endNoHyper}}%
      }{\parbox[t]{\columnwidth}{\normalfont\small\XC@tips}}%
    \fi%
  }%\protect\endNoHyper}}

\let\XC@oldcite\citep  % Save \citep (in the case if it becomes redefined)

%    \end{macrocode}
% \end{macro}
%
%%%%%%%%%%%%%%%%%%%%%%%%%%%%%%%%%%%%%%%%%%%%%%%%%%%%%%%%%%%%%%%%%%%%%%%%%%%%%%%%
% \begin{macro}{\xpdfcite}
%
% A wrapper for the combined |\pdftooltip| and |\citep|.\\
% It has the same syntax as |\citep|.
%
%    \begin{macrocode}

\let\xpdfcite\XC@citetp
\robustify{\xpdfcite}

\AtBeginDocument{\nobibliography*} % Necessary to get bibentries.

%    \end{macrocode}
% \end{macro}
%
%%%%%%%%%%%%%%%%%%%%%%%%%%%%%%%%%%%%%%%%%%%%%%%%%%%%%%%%%%%%%%%%%%%%%%%%%%%%%%%%
% \Finale
